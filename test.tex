\documentclass{ctexart}

\usepackage{amsfonts}
\usepackage{amsmath}
\usepackage{amsthm}
\newtheorem{theorem}{Theorem}
\newtheorem{lemma}{Lemma}
\newtheorem{definition}{Definition}

\DeclareMathOperator*{\rank}{rank}
\DeclareMathOperator*{\diag}{diag}

\usepackage{pdfpages}

\begin{document}
\section{奇异值分解}
    为了引入奇异值的概念,我们先证明两个引理.
    \begin{lemma}
        设$A\in\mathbb{C}^{m\times n}$,则
        $$\rank(A^HA)=\rank(AA^H)=\rank(A)$$
    \end{lemma}
    \begin{proof}
        利用$Ax=0$与$A^HAx=0$同解以及$Ax=0$与$AA^Hx=0$同解立刻可得.
    \end{proof}
    \begin{lemma}
        设$A\in\mathbb{C}^{m\times n}$,则

        (1)$A^HA$与$AA^H$的特征值均为非负实数;

        (2)$A^HA$与$AA^H$的非零特征值相同,并且非零特征值的个数(重特征值按重数计算)等于$\rank(A)$.
    \end{lemma}
    \begin{proof}
        (1)设$\lambda$是$A^HA$的任一特征值,$x\neq0$为相应的特征向量,则$$A^HAx=\lambda x$$
        因为$A^HA$是Hermite矩阵,所以$\lambda$是实数,并且$$\lambda x^Hx=x^HA^HAx=(Ax)^HAx\geq0$$
        由于$x^x>0$,所以$\lambda\geq0$.
        
        同理可证$AA^H$的特征值军费非负实数.
        
        (2)由线性代数知识,易知方阵$AB$与方阵$BA$的非零特征值相同,则$A^HA$与$AA^H$的非零特征值相同,并且它们的非零特征值个数为$$r=\rank(A^HA)=\rank(AA^H)=\rank(A)$$
    \end{proof}
    \newpage
    \begin{theorem}
        设$A$是$m\times n$矩阵,且$\rank(A)=r$,则存在$m$阶酉矩阵$V$和$n$阶酉矩阵$U$使得
        $$V^HAU=
        \begin{pmatrix}
            \Sigma & 0 \\
            0 & 0
        \end{pmatrix}
        $$
        其中$\Sigma=\diag(\sigma_1,\cdots,\sigma_r)$,且$\sigma_1\geq\cdots\geq\sigma_r>0$.
    \end{theorem}
    \begin{proof}
        因为$\rank(A)=r$,由Lemma2可设$A^HA$的特征值是
        $$\sigma^2_1\geq\cdots\geq\sigma^2_r>0,\sigma^2_{r+1}=\cdots=\sigma^2_{n}=0$$
        因为$A^HA$是Hermite矩阵,由谱分解定理可知存在$n$阶酉矩阵$U$使得
        $$U^HA^HAU=
        \begin{pmatrix}
            \Sigma^2 & 0 \\
            0 & 0
        \end{pmatrix}
        $$
        记$U=[U_1,U_2]$,其中$U_1$是$n\times r$矩阵,上式可改写为
        $$
        A^HA[U_1,U_2]=[U_1,U_2]
        \begin{pmatrix}
            \Sigma^2 & 0 \\
            0 & 0
        \end{pmatrix}
        $$
        则有
        $$A^HAU_1=U_1\Sigma^2,A^HAU_2=0$$

        记$V=[V_1,V_2]$,其中$V_1$是$m\times r$矩阵,$V_2$是$m\times(m-r)$矩阵.令
        $$V_1=AU_1\Sigma^{-1}$$
        则
        $$V^H_1V=(\Sigma^{-1})^HU^H_1A^HAU_1\Sigma^{-1}=I_r$$
        那么我们可以找到一个$V_2$使得$V=[V_1,V_2]$是酉矩阵,则
        $$V^H_2AU_1=V^H_2V_1\Sigma=0$$
        那么
        $$V^HAU=
        \begin{pmatrix}
            V^H_1 \\
            V^H_2
        \end{pmatrix}A[U_1,U_2]=
        \begin{pmatrix}
            V^H_1AU_1 & V^H_1AU_2 \\
            V^H_2AU_1 & V^H_2AU_2
        \end{pmatrix}=
        \begin{pmatrix}
            \Sigma & 0 \\
            0 & 0
        \end{pmatrix}
        $$
    \end{proof}
\section{矩阵级数}
    \begin{definition}
        设有矩阵序列$\{A^{(k)}\}$,若当$k\to+\infty$时,有$a_{ij}^{(k)}\to a_{ij}$,则称$\{A^{(k)}\}$收敛,并将$A=(a_{ij})$称为$\{A^{(k)}\}$的极限,
        记为$\lim_{k\to+\infty}A^{(k)}=A$.不收敛的矩阵序列称为发散的.
    \end{definition}
    \begin{lemma}
        Definition 1与矩阵范数意义下的收敛性是等价的
    \end{lemma}
\end{document}